% vim: ft=tex fdm=marker et sts=2 sw=2 cole=0
%
% vitae.tex -- a minimalistic LaTeX template for resumes
%
% These days most people read resumes on their computers.  Therefore,
% using sans-serif fonts makes more sense.  Apart from this, the design
% should be kept as minimalistic as possible, should not use LaTeX
% packages that are not part of the standard TeXLive distribution, and
% one should be able to distribute the entire resume as a single LaTeX
% file.
%
%                                 ***
%
% Written and modified between 2014 and 2021 by Manu Mannattil
%
% Usage, modification, and distribution of this document, for any
% purpose, with or without attribution, is permitted provided the
% author's personal/professional details are removed.
%

% Preamble {{{1
% -------------

\documentclass[10pt]{article}

\newcommand{\fname}{Manu Mannattil}             % Set your full name here.
\newcommand{\iname}{M.~Mannattil}               % Set your intials + last name here.
\newcommand{\bname}{\underline{Manu Mannattil}} % Set your name as you'd like it to appear in bibliographies.

\usepackage[T1]{fontenc}
\usepackage[utf8]{inputenc}
\usepackage{microtype}
\usepackage{ragged2e}

% Required to break up tables across multiple pages.
\usepackage{longtable}
\setlength\LTleft{0pt}  % flush left
\setlength\LTpre{0pt}   % space before table

\usepackage[
  top=1.1in,
  bottom=0.85in,
  left=1.2in,
  right=1.2in,
  headsep=0.25in,
% showframe
]{geometry}

\usepackage[x11names]{xcolor}
\usepackage[
  allcolors=Blue4,
  colorlinks=true,
  pdftex,
  pdfauthor={\fname},
  pdftitle={Curriculum Vitae: \fname},
  pdfsubject={Curriculum Vitae},
  pdfkeywords={curriculum vitae, resume, vitae}
]{hyperref}
\urlstyle{same}

% Custom commands.
\providecommand{\email}[1]{\href{mailto:#1}{#1}}
\providecommand{\doi}[2]{\href{https://dx.doi.org/#1}{#2}}

% Beautiful Helvetica throughout.
\usepackage[scale=0.92]{tgheros}
\usepackage{tgtermes}
\renewcommand{\familydefault}{\sfdefault}
\usepackage{sansmath}
\sansmath

% Customize page headers/footers.
\usepackage{fancyhdr}
\pagestyle{fancy}
\rhead{\iname}
\lhead{\it{Curriculum Vit\ae}}
\renewcommand{\headrulewidth}{0.75pt}
\fancyfoot[C]{\thepage}
\thispagestyle{plain}

% Section heading formatting.
\usepackage{titlesec}
\titleformat{\section}{\bfseries}{\thesection. }{0pt}{\MakeUppercase}
\titlespacing*{\section}{0pt}{10pt}{10pt}
\titleformat{\subsection}{\bfseries}{\thesubsubsection. }{0pt}{}
\titlespacing*{\subsection}{0pt}{8pt}{8pt}

% For getting Git commit info.
\usepackage{gitinfo2}

% For setting spacing in lists.
\usepackage{enumitem}

\setlength\parindent{0em}
\widowpenalty10000
\clubpenalty10000

\begin{document}

% Name and Contact Details {{{1
% -----------------------------

{\bfseries\large\fname}
\medskip

\begin{minipage}[t]{0.45\textwidth}
  Department of Physics\\
  Syracuse University\\
  Syracuse, NY 13244
\end{minipage}
%
\begin{minipage}[t]{0.45\textwidth}
  Phone: (+1)~315-515-7373\\
  Email: \email{manu.mannattil@posteo.net}
\end{minipage}\\

% Personal Information {{{1
% -------------------------

\section*{Personal Information}

Born on August 13, 1991 in Palakkad, Kerala; Indian citizen.

% Education {{{1
% --------------

\section*{Education}

\begin{tabular}{@{}ll}
  \emph{Aug.~2017 -- present}   & Syracuse University, New York, USA: Ph.D. in Physics\\
%                               & GPA: 3.958/4.000; Adviser: \href{http://cdsantan.expressions.syr.edu}{Chris Santangelo}\\
  \emph{Jul.~2009 -- Jul. 2014} & Indian Institute of Technology Kanpur, Uttar Pradesh, India: Int.~M.Sc.~in Physics\\
%                               & GPA: 6.00/10.0; Adviser: \href{https://home.iitk.ac.in/~sagarc}{Sagar Chakraborty}\\
% \emph{Jun.~2007 -- Mar. 2009} & Bhavan's Vidya Mandir, Poochatty, Thrissur, Kerala: Indian Senior School Certificate\\
%                               & Percentage: 88.4\%\\
% \emph{May 2005 -- Mar. 2007}  & Hari Sri Vidya Nidhi School, Thrissur, Kerala: Indian Certificate of Seconday Education\\
%                               & Percentage: 93.4\%
\end{tabular}

% Employment History {{{1
% -----------------------

\section*{Employment History}

\begin{tabular}{@{}ll}
  \emph{May 2020 -- present}     & Research Assistant, Department of Physics, Syracuse University, USA\\
  \emph{Aug.~2017 -- May 2020}   & Teaching Assistant, Department of Physics, Syracuse University, USA\\
  \emph{Sep.~2014 -- Oct. 2015}  & Project Associate, Nonlinear Dynamical Systems Group,\\
                                 & Indian Institute of Technology Kanpur, India
\end{tabular}

% Publications {{{1
% -----------------

\section*{Publications}

% \subsection*{Refereed Journal Papers}

\begin{enumerate}[itemsep=0pt,leftmargin=20pt]
  \item[4.] \bname, Ambrish Pandey, Mahendra K.~Verma, and Sagar Chakraborty, ``On the applicability of low-dimensional models for convective flow reversals at extreme Prandtl numbers,'' \doi{10.1140/epjb/e2017-80391-1}{Eur.~Phys.~J.~B~\textbf{90}, 259~(2017)}.
  \item[3.] \bname, Himanshu Gupta, and Sagar Chakraborty, ``Revisiting Evidence of Chaos in X-ray Light Curves: The Case of GRS~1915+105,'' \doi{10.3847/1538-4357/833/2/208}{Astrophys.~J.~\textbf{833}, 208~(2016)}.
  \item[2.] Aditya Tandon, Malte Schr\"{o}der, \bname, Marc Timme, and Sagar Chakraborty, ``Synchronizing noisy nonidentical oscillators by transient uncoupling,'' \doi{10.1063/1.4959141}{Chaos \textbf{26}, 094817~(2016)}.
  \item[1.] Malte Schr\"{o}der, \bname, Debabrata Dutta, Sagar Chakraborty, and Marc Timme, ``Transient Uncoupling Induces Synchronization,'' \doi{10.1103/PhysRevLett.115.054101}{Phys.~Rev.~Lett.~\textbf{115}, 054101~(2015)}.
\end{enumerate}

% \subsection*{Scientific Software}

% \begin{enumerate}[itemsep=0pt,leftmargin=20pt]
%   \item[1.] \bname, ``NoLiTSA: A Python module for nonlinear time series analysis'' (2015--2017)\\
%     Source: \url{https://github.com/manu-mannattil/nolitsa}
% \end{enumerate}

% \subsection*{Submissions and Works in Progress}

% Awards and Honors {{{1
% ----------------------

\section*{Awards and Honors}

\begin{tabular}{@{}ll}
  \emph{May 2020}  & College of Arts \& Sciences Award, Syracuse University\\
  \emph{Apr.~2018} & Fellowship from the Henry Levinstein fund, Department of Physics, Syracuse University\\
  \emph{Mar.~2011} & Kishore Vaigyanik Protsahan Yojana, Department of Science and Technology, Government of India
\end{tabular}

% Talks {{{1
% ----------

\section*{Talks}

\begin{tabular}{@{}ll}
  \emph{Mar.~2021} & ``Thermal Fluctuations of Singular Mechanical Networks'' at APS Virtual March Meeting 2021\\
  \emph{Aug.~2015} & ``Introduction to Nonlinear Time Series Analysis'' at Modelling and Simulation Lab,\\ & Indian Institute of Technology Kanpur, Uttar Pradesh, India
\end{tabular}

% Service {{{1
% ------------

\section*{Service to Profession}

Manuscript reviewer for \emph{Physics of Fluids}.

% Teaching {{{1
% -------------

\section*{Teaching Experience}

\subsection*{Syracuse University}

Instructor for PHY 212 (Electricity \& Magnetism); Teaching Assistant for PHY 215 (Honors Mechanics), PHY 211 (General Mechanics), PHY 222 (Electricity \& Magnetism Lab), and AST 101 (Introductory Astronomy).

% \begin{longtable}{@{}rl}
%   \emph{Spring 2020} & Teaching Assistant, PHY 212: General Physics II (Electricity \& Magnetism)\\
%   \emph{Fall 2019}   & Teaching Assistant, AST 101: Our Corner of the Universe\\
%   \emph{Summer 2019} & Course Instructor, PHY 212: General Physics II (Electricity \& Magnetism)\\
%   \emph{Spring 2019} & Teaching Assistant, PHY 222: General Physics II Lab (Electricity \& Magnetism)\\
%   \emph{Fall 2018}   & Teaching Assistant, PHY 211: General Physics I (Mechanics)\\
%   \emph{Fall 2018}   & Teaching Assistant, PHY 215: General Physics I for Majors (Mechanics)\\
%   \emph{Summer 2018} & Teaching Assistant, AST 101: Our Corner in the Universe\\
%   \emph{Spring 2018} & Teaching Assistant, PHY 212: General Physics II (Electricity \& Magnetism)\\
%   \emph{Fall 2017}   & Teaching Assistant, AST 101: Our Corner in the Universe
% \end{longtable}


% % Schools and Workshops Attended {{{1
% % -----------------------------------

% \section*{Schools and Workshops Attended}

% \begin{tabular}{@{}ll}
%   \emph{Aug.~2011} & 5th Asian Science Camp: KAIST, Daejeon, South Korea\\
%   \emph{Jun.~2010} & National Initiative for Undergraduate Science: Homi Bhabha Centre for Science Education, India
% \end{tabular}

% Miscellaneous {{{1
% ------------------

\section*{Miscellaneous}

\begin{tabular}{@{}p{1in}p{5in}}
  \emph{Languages}   & English (fluent), Hindi (basic), and Malayalam (native)\\
  \emph{Programming} & Extensive experience writing modular software following the ``one-thing-well'' philosophy in various programming languages including Python, C, Vim script, AWK, and various shells (Bash, POSIX sh, etc.)
\end{tabular}

% Revision Information {{{1
% -------------------------

\bigskip
\begin{center}
  \hypersetup{hidelinks}\color{gray}
  Git commit \href{https://github.com/manu-mannattil/vitae/tree/\gitHash}{\texttt{\gitAbbrevHash}}; \gitAuthorIsoDate
\end{center}

\end{document}
